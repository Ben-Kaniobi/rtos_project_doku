%%%%%%%%%%%%%%%%%%%%%%%%%%%%%%%%%%%%%%%%%%%%%%%%%%%%%%%%%%%%%%%%%%%%%%%%%%%%%%%
% Titel:   Bericht - Pakete
% Autor: Simon Grossenbacher  
% Datum:   27.09.2013
% Version: 1.0.0
%%%%%%%%%%%%%%%%%%%%%%%%%%%%%%%%%%%%%%%%%%%%%%%%%%%%%%%%%%%%%%%%%%%%%%%%%%%%%%%
%
%:::Change-Log:::
% Versionierung erfolgt auf folgende Gegebenheiten: -1. Release Versionen
%                                                   -2. Neue Kapitel
%                                                   -3. Fehlerkorrekturen
%
% 0.0.0       Erstellung der Datei
%
%:::Hinweis:::
% Indexerstellung: makeindex -s report.ist report.idx
%   Umlaute m�ssen separat behandelt werden!
%%%%%%%%%%%%%%%%%%%%%%%%%%%%%%%%%%%%%%%%%%%%%%%%%%%%%%%%%%%%%%%%%%%%%%%%%%%%%%%

% Fix f�r KOMA-Script
\usepackage{scrhack}

% Sprach-Optionen
\usepackage[ngerman]{babel}         % Neue deutsche Rechtschreibung
\usepackage[T1]{fontenc}            % Richtige Worttrennung
%\usepackage[applemac]{inputenc}    % Mac - load extended character set (ISO 8859-1)
%\usepackage[latin1]{inputenc}      % Unix/Linux - load extended character set (ISO 8859-1)
\usepackage[ansinew]{inputenc}      % Windows - load extended character set (ISO 8859-1)
%\usepackage[utf8]{inputenc}        % UTF-8 encoding

% Zeilenabstand
\usepackage{setspace}

% Mehr Tabellenoptionen
\usepackage{tabularx}
\usepackage{longtable}

% Listen
\usepackage{enumitem}

% Besserer Flattersatz
\usepackage{ragged2e}

% Gleiten verhindern
\usepackage{float}
\usepackage{placeins}

%Ueberschriften anpassen
\usepackage{titlesec} 

% Farben
\usepackage{color}
\usepackage{colortbl} %F�r farbige Tabellen

% PDF zu Dokument hinzufuegen
\usepackage[final]{pdfpages}

% Grafiken verwalten
\usepackage{graphicx}
\usepackage[absolute]{textpos}

% Zeichnen
%\usepackage{pst-pdf}
%\usepackage{pst-all}

% Listnings verwalten
\usepackage{listings}

% Kopf- Fusszeile (Optionen m�ssen direkt �bergeben werden)
\usepackage[automark,           % Automatisches aktualisieren der Chapter-Titel
%			headsepline,        % Linie Kopfzeile
%			footsepline,        %Linie fusszeilezeile
%			markuppercase,
			plainfootsepline    % Plain-Style auch mit Linie versehen
			]{scrpage2}

% Flexible Argumente bei Funktionen
\usepackage{xargs}

% Erweiterte Steuerfunktionen
\usepackage{ifthen}

% Index f�r Stichwortverzeichnis
\usepackage{makeidx}

% Index f�r Literaturverzeichnis
\usepackage[babel,german=quotes]{csquotes}
%\usepackage[backend=biber,style=numeric,defernumbers=true,sorting=nyt]{biblatex}
\usepackage[backend=bibtex,defernumbers=true]{biblatex}
\bibliography{bibliography}
\defbibheading{lit}{\section{Literatur}}
\defbibheading{pic}{\section{Abbildungen}}

% Zusaetzliche Symbole direkt im Text
\usepackage{textcomp}
\usepackage{amssymb}

% Einheit kontrolliert eingeben
% units -> Sch�ne Darstellung:
% Z.B. \unitfrac[1.2]{m}{s} ergibt 1.2m/s (wobei m/s wie ein Zeichen)
% siunitx -> SI-Einheiten:
% Z.B. \SI{1.2}{\meter\per\second} ergibt 1,2 ms-1
% (wobei -1 hochgestellt, L�cke vor Einheit halb so gross und Komma automatisch)
\usepackage{units}
\usepackage{siunitx}
\sisetup{locale=DE}

% Dynamische Datumsausgabe
\usepackage[german]{isodate}

% Zusaetzlich Mathemtiksymbole
\usepackage{amsmath}
\usepackage{mathtools}

% Besser Handling von internen Countern und Berechnungen
\usepackage{calc}

% TODOs anbringen am Rand
\usepackage{todonotes}

% Hyperlinks (Muss das letzte geladene Paket sein)
\usepackage[bookmarks=true, %Verzeichnis generieren
            bookmarksopen=true, %Verzeichnis �ffnen
            bookmarksopenlevel=3, %Tiefe der Verzeichnis�ffnung
            unicode=false, %non-Latin Zeichen
            pdftoolbar=true, %PDF-Viewer Toolbar
            pdfmenubar=true, %PDF-Viewer Men�?
            pdffitwindow=true, %Fenster an Seite anpassen beim �ffnen
            pdftitle={\Titel}, %Titel
            %pdfauthor={\Autor1, \Autor2, \Autor3}, %Autor
            pdfsubject={\Uebertitel}, %Thema
%            pdfcreator={\Autor1, \Autor2, \Autor3}, %Ersteller des Dokuments
%            pdfproducer={\Autor1, \Autor2 \Autor3}, %Produzent des Dokuments
            pdfnewwindow=true, %Links in neuem Fenster
            colorlinks=true, %false: Boxen-Links; true: Farben-Links
            linkcolor=black, %Farbe von internen Links
            citecolor=black, %Farbe von Links zu Bibliography
            filecolor=magenta, %Farbe von Links zu Dateien
            urlcolor=blue %Farbe von externen Links
            ]{hyperref}
