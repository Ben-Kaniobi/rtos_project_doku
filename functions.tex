%%%%%%%%%%%%%%%%%%%%%%%%%%%%%%%%%%%%%%%%%%%%%%%%%%%%%%%%%%%%%%%%%%%%%%%%%%%%%%%
% Titel:   Bericht - Funktionen
% Autor:   Simon Grossenbacher  
% Datum:   27.09.2013
% Version: 1.0.0
%%%%%%%%%%%%%%%%%%%%%%%%%%%%%%%%%%%%%%%%%%%%%%%%%%%%%%%%%%%%%%%%%%%%%%%%%%%%%%%
%
%:::Change-Log:::
% Versionierung erfolgt auf folgende Gegebenheiten: -1. Release Versionen
%                                                   -2. Neue Kapitel
%                                                   -3. Fehlerkorrekturen
%
% 0.0.0       Erstellung der Datei
%
%:::Hinweis:::
% Indexerstellung: makeindex -s report.ist report.idx
%   Umlaute m�ssen separat behandelt werden!
%%%%%%%%%%%%%%%%%%%%%%%%%%%%%%%%%%%%%%%%%%%%%%%%%%%%%%%%%%%%%%%%%%%%%%%%%%%%%%%

%%%%%%%%%%%%%%%%%%%%%%%%%%%%%%%%%%%%%%%%%%%%%%%%%%%%%%%%%%%%%%%%%%%%%%%%%%%%%%%
% Text tiefstellen
% param #1 Text
%%%%%%%%%%%%%%%%%%%%%%%%%%%%%%%%%%%%%%%%%%%%%%%%%%%%%%%%%%%%%%%%%%%%%%%%%%%%%%%
\newcommand{\low}[1]{\textsubscript{#1}}



%%%%%%%%%%%%%%%%%%%%%%%%%%%%%%%%%%%%%%%%%%%%%%%%%%%%%%%%%%%%%%%%%%%%%%%%%%%%%%%
% Text hochstellen
% param #1 Text
%%%%%%%%%%%%%%%%%%%%%%%%%%%%%%%%%%%%%%%%%%%%%%%%%%%%%%%%%%%%%%%%%%%%%%%%%%%%%%%
\newcommand{\high}[1]{\textsuperscript{#1}}



%%%%%%%%%%%%%%%%%%%%%%%%%%%%%%%%%%%%%%%%%%%%%%%%%%%%%%%%%%%%%%%%%%%%%%%%%%%%%%%
% Schrift anpassen
% param #1 Schriftfamilie: ptm  Times
%                          phv 	Helvetica
%                          pcr 	Courier
%                          pbk 	Bookman
%                          pag 	Avant Garde
%                          ppl 	Palatino
%                          pch 	Charter
%                          pnc 	New Century Schoolbook
%                          put 	Utopia
% param #2 Strichdicke/Zeichenbreite: b  bold
%                                     m  medium
% param #3 Schriftform: n   normal
%                       it  italic
%                       sl  slanted
%                       sc  small caps
%                       ui  upright italic
% note {\changefont{#1}{#2}{#3} Hallo Welt} Teilbereich aendern
%%%%%%%%%%%%%%%%%%%%%%%%%%%%%%%%%%%%%%%%%%%%%%%%%%%%%%%%%%%%%%%%%%%%%%%%%%%%%%%
\newcommand{\changefont}[3]{\fontfamily{#1} \fontseries{#2} \fontshape{#3} \selectfont}



%%%%%%%%%%%%%%%%%%%%%%%%%%%%%%%%%%%%%%%%%%%%%%%%%%%%%%%%%%%%%%%%%%%%%%%%%%%%%%%
% Zum Beispiel abkuerzen ohne Trennung
% param none
%%%%%%%%%%%%%%%%%%%%%%%%%%%%%%%%%%%%%%%%%%%%%%%%%%%%%%%%%%%%%%%%%%%%%%%%%%%%%%%
\newcommand{\zB}{z.~B.}



%%%%%%%%%%%%%%%%%%%%%%%%%%%%%%%%%%%%%%%%%%%%%%%%%%%%%%%%%%%%%%%%%%%%%%%%%%%%%%%
% Formeleintrag
% param #1 Formel
% param #2 Parameter Beschreibung im Tabellensyntax
% param #3 Formel-Label (optional)
%%%%%%%%%%%%%%%%%%%%%%%%%%%%%%%%%%%%%%%%%%%%%%%%%%%%%%%%%%%%%%%%%%%%%%%%%%%%%%%
%Savebox fuer Parameterbeschreibung
\newsavebox{\myendhook} % for the tabulars
\makeatletter
\def\tagform@#1{{(\maketag@@@{\ignorespaces#1\unskip\@@italiccorr)}
\makebox[0pt][r]{% after the equation number
\makebox[0.5\textwidth][l]{\usebox{\myendhook}}%
}%
\global\sbox{\myendhook}{}% clear box content
}}
\makeatother 
%Kommando definieren
\newcommandx{\formula}[3][3=\empty,usedefault]{
    \sbox{\myendhook}{%
        \begin{footnotesize}%
            \begin{tabular}{@{}l >{\RaggedRight}p{.33\textwidth}}%0.33
                #2%\ifthenelse{\equal{#2}{\empty}}{ }{#2}
            \end{tabular}
        \end{footnotesize}}
    %
    \begin{equation}
        \ifthenelse{\equal{#3}{\empty}}
        {}
        {\label{#3}}%anstatt equation	
        \boxed{\begin{split}#1\end{split}}
        %\begin{split}#1\end{split}
    \end{equation} %anstatt equation
}



%%%%%%%%%%%%%%%%%%%%%%%%%%%%%%%%%%%%%%%%%%%%%%%%%%%%%%%%%%%%%%%%%%%%%%%%%%%%%%%
% Bildereintrag
% param #1 Pfad zum Bild
% param #2 Groesse: z.B. scale=0.5
% param #3 Bildunterschrift (optional)
% param #4 Bildunterschrift im Abbildungsverzeichnis (optional)
% param #5 Bildlabel (optional)
%%%%%%%%%%%%%%%%%%%%%%%%%%%%%%%%%%%%%%%%%%%%%%%%%%%%%%%%%%%%%%%%%%%%%%%%%%%%%%%
\newlength{\imagewidth}
\newlength{\imageheight}
\newcommandx{\image}[5][3=\empty,4=\empty:,5=\empty,usedefault]{
    \begin{figure}[H]%H htbp
        \centering
        \settowidth{\imagewidth}{\includegraphics[#2]{#1}}
        \settoheight{\imageheight}{\includegraphics[#2]{#1}}
        \ifdim\imagewidth<\textwidth
            \ifdim\imageheight<\textheight
                \includegraphics[#2]{#1}
            \else
                \includegraphics[height=\textheight]{#1}
            \fi
        \else
            \ifdim\imageheight<\textheight
                \includegraphics[width=\textwidth]{#1}
            \else
                \setlength{\imageheight}{\imageheight-\textheight}
                \setlength{\imagewidth}{\imagewidth-\textwidth}
                \ifdim\imageheight<\imagewidth
                    \includegraphics[width=\textwidth]{#1}
                \else
                    \includegraphics[height=\textheight]{#1}
                \fi
            \fi
        \fi
        \ifthenelse{\equal{#5}{\empty}}
        {
            \ifthenelse{\equal{#3}{\empty}}{}{\caption{#3}}
            \ifthenelse{\equal{#4}{\empty}}{}{\label{#4}}
        }
        {
            \caption[#4]{#3}
            \label{#5}
        }
    \end{figure}
}



%%%%%%%%%%%%%%%%%%%%%%%%%%%%%%%%%%%%%%%%%%%%%%%%%%%%%%%%%%%%%%%%%%%%%%%%%%%%%%%
% Bildereintrag fuer Tabellen
% param #1 Pfad zum Bild
% param #2 Groesse: z.B. scale=0.5
%%%%%%%%%%%%%%%%%%%%%%%%%%%%%%%%%%%%%%%%%%%%%%%%%%%%%%%%%%%%%%%%%%%%%%%%%%%%%%%
\newlength{\myx} % Variable zum Speichern der Bildbreite
\newlength{\myy} % Variable zum Speichern der Bildh�he
\newcommand{\imagetotab}[2][\relax]{%
% Abspeichern der Bildabmessungen
\settowidth{\myx}{\includegraphics[{#1}]{#2}}%
\settoheight{\myy}{\includegraphics[{#1}]{#2}}%
% das eigentliche Einf�gen
\parbox[c][1.1\myy][c]{\myx}{%
\includegraphics[{#1}]{#2}}%
}
