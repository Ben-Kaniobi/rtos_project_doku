%%%%%%%%%%%%%%%%%%%%%%%%%%%%%%%%%%%%%%%%%%%%%%%%%%%%%%%%%%%%%%%%%%%%%%%%%%%%%%%
% Titel:   Abstract
% Autor:   S. Grossenbacher
% Datum:   27.09.2013
% Version: 1.0.0
%%%%%%%%%%%%%%%%%%%%%%%%%%%%%%%%%%%%%%%%%%%%%%%%%%%%%%%%%%%%%%%%%%%%%%%%%%%%%%%

%:::Change-Log:::
% Versionierung erfolgt auf folgende Gegebenheiten: -1. Stelle Semester
%                                                   -2. Stelle neuer Inhalt
%                                                   -3. Fehlerkorrekturen
%
% 1.0.0       Erstellung der Datei

%%%%%%%%%%%%%%%%%%%%%%%%%%%%%%%%%%%%%%%%%%%%%%%%%%%%%%%%%%%%%%%%%%%%%%%%%%%%%%%
%\chapter*{Abstract}
%Im Unterricht des Moduls \textit{Echtzeit-Betriebssysteme} der Berner Fachhochschule wird im Rahmen eines Projektes ein \textit{RTOS} in Betrieb genommen.\par
%%
%Das Projekt dient dazu, das gelernte Wissen aus dem zugeh�rigen Vorlesungsmodul anzuwenden und zu verinnerlichen. Die Dokumentation dient dazu, die angestellten �berlegungen aufzuzeigen. 

\chapter*{Vorwort}
Einmal mehr im Werdegang eines angehenden Ingenieurs steht die Umsetzung eines Projektes zwischen uns und einer guten Note. Einmal mehr macht diese gute Note nur einen kleinen Teil der gesamten Bewertung des Moduls aus. Und ebenfalls einmal mehr macht die Dokumentation nur einen kleinen Teil der Bewertung des Projektes aus. Aus diesem Grund soll diese Dokumentation unter anderem dazu dienen, dass wir uns im Kurzfassen �ben k�nnen. Es soll auf abrundendes Prosa verzichtet werden, wo immer dadurch sinnvoll Zeit gespart werden kann.\par
%
Diese Dokumentation richtet sich demnach an den bewertenden Dozenten, welcher sowohl die Entwicklungsschritte und die eingesetzten Methoden als auch die Sprache an sich sehr gut beherrscht. Sie soll aufzeigen, was wir uns �berlegt haben, jedoch auf alles allgemeine oder unn�tige verzichten.\par
%
Um den Projekt-Overhead durch die Dokumentation m�glichst klein zu halten, verzichten wir auf das Anf�gen von Stichwortverzeichnissen, ausf�hrlichen Einleitungen, �berm�ssiger Bebilderung und so �hnlichem.\par
%
Trotzdem w�nschen wir dem Leser viel Vergn�gen!
