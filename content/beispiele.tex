%%%%%%%%%%%%%%%%%%%%%%%%%%%%%%%%%%%%%%%%%%%%%%%%%%%%%%%%%%%%%%%%%%%%%%%%%%%%%%%
% Titel:   Beispiele
% Autor:   Simon grossenbacher
% Datum:   22.09.2013
% Version: 1.0.1
%%%%%%%%%%%%%%%%%%%%%%%%%%%%%%%%%%%%%%%%%%%%%%%%%%%%%%%%%%%%%%%%%%%%%%%%%%%%%%%
%
%:::Change-Log:::
% Versionierung erfolgt auf folgende Gegebenheiten: -1. Release Versionen
%                                                   -2. Neue Kapitel
%                                                   -3. Fehlerkorrekturen
%
% 1.0.0       Erstellung der Datei
% 1.0.1       Bilder im Abschnitt "Verlinkungen" hinzugef�gt
%%%%%%%%%%%%%%%%%%%%%%%%%%%%%%%%%%%%%%%%%%%%%%%%%%%%%%%%%%%%%%%%%%%%%%%%%%%%%%%    
\chapter{Beispiele}\label{ch:beispiele}
    Das ist ein Beispiel-Kapitel.
    %
    %
    %Ueberschrift 1
    \section{�berschrift 1}\label{s:uebschrift_1}
        Das ist eine grosse �berschrift!\par
        N�chster Absatz.
        %
        %Ueberschrift 2
        \subsection{�berschrift 2}\label{ss:ueberschrift_2}
            Dies ist eine kleinere �berschrift!
            %
            \subsubsection{�berschrift 3}\label{sss:uberschrift_3}
                Noch eine kleinere �berschrift!
    %
    %
    %Aufz�hlungen
    \section{Aufz�hlungen}\label{s:aufzaehlungen}
        Es gibt mehrere M�glichkeiten:
        \begin{itemize}
            \item Hallo
            \item Tsch�ss
        \end{itemize}
        \begin{enumerate}
            \item Eis
            \item Zw�i
        \end{enumerate}
        \begin{description}
            \item[Ich] Du Ich Du 
            \item[Er] Sie Er Sie
        \end{description}
    %
    %
    %Formeln
    \section{Formeln}\label{s:formeln}
        Hier bietet sich die Funktion \texttt{formula} an.
        \formula{
            U=R\cdot I
        }{
            $U$ & Spannung in Volt\\
            $R$ & Widerstand\\
            $I$ & Strom}[eq:uri]
        Es k�nnen auch mehrere Formeln zusammengefasst werden (man beachte das \textbf{\&} f�r die Ausrichtung)!
        \formula{
            U&=R\cdot I\\
            R&= \frac{U}{I}
        }{
            $U$ & Spannung in Volt\\
            $R$ & Widerstand\\
            $I$ & Strom}[eq:uri2]
    %        
    %
    %Bilder
    \section{Bilder}\label{s:bilder}
        Bilder k�nnen mit der Funktion \texttt{image} ein gef�gt werden.
        \image{content/image/bfh_logo}{scale=1}[Unterschrift f�r unter dem Bild \cite{pic:bfh}][Bildbeschreibung f�r im Abbildungsverzeichnis][img:bfh]
    %
    %
    %Tabellen
    \section{Tabellen}\label{s:tabellen}
        Tabellen in \LaTeX sind ziemlich umst�ndlich. Daher empfiehlt sich ein entsprechendes Plugin in der Office-Umgebung, die ein \LaTeX -Export erm�glichen. 
        \begin{table}[htbp]
             \centering
             \begin{tabular}{|l|l|l|} 
                 \hline
                 \rowcolor{bfhblue}
                 \textcolor{white}{Spalte 1} & \textcolor{white}{Spalte 2} & \textcolor{white}{Spalte 3}\\
                 \hline
                 Ich & bin & da \\
                 \hline
                 \multicolumn{2}{|l|}{Zwei Spalten vereinen} & Das geht!\\
                 \hline
             \end{tabular}
             \caption{Messmittelliste der Messung Messleitung}
             \label{tab:tabelle}  
        \end{table}
        \begin{itemize}
            \item Falls Bilder in Tabellen n�tig sein sollten, Befehl \texttt{imagetotab} verwenden      
            \item Tabellen �ber mehrere Seiten sind mit der \texttt{longtable} m�glich
        \end{itemize}
    %
    %
    %Code
    \section{Code}\label{s:code}
        \subsection{ANSI C}\label{ss:c}
            \begin{lstlisting}[style=C,caption={Hallo Welt},label={list:hallowelt}]
printf("Hallo Welt"); //Dummy Funktion
            \end{lstlisting}  
        %
        % 
        \subsection{MATLAB}\label{ss:matlab}
            \begin{lstlisting}[style=Matlab,caption={Sinnlos},label={list:sinnlos}]
close all
clear all
t = [0:1000];
plot(t,t); %Dummy-Plot
            \end{lstlisting}  
    %
    %
    %Verlinkungen
    \section{Verlinkungen}\label{s:verlinkungen}
        \begin{description}
            \item[Fussnoten] Fussnoten werden mit dem Kommando \texttt{footnote} erstellt\footnote{Ich bin eine normale Fussnote}
            \item[Verweise] Es kann auf alle Labels verwiesen werden. Dazu dienen die Befehle \texttt{ref} und \texttt{pageref}: Das Kapitel \ref{s:verlinkungen} befindet sich auf Seite \pageref{s:verlinkungen}
            \item[Quellen] Um auf Quellen zu verweisen, muss das Label bekannt sein (wird in der Datei \texttt{bibliography.bib} festgelegt). Anschliessend muss im \LaTeX~Editor \texttt{BibTex} ausgef�hrt werden. Daraufhin kann der Befehl \texttt{cite} eingesetzt werden \cite{lit:sus}. Evtl. muss das Dokument mehrmals compiliert werden. F�r das bessere Verwalten der Quellen, kann bei Bedarf auch ein entsprechender Editor verwendet werden\footnote{z.B. JabRef \url{http://jabref.sourceforge.net/}}.
            \image{content/image/bibtex}{scale=.5}[Literaturverzeichnis erstellen]
            \image{content/image/bib}{scale=.5}[Ausschnitt aus der \texttt{bibliography.bib} Datei]
            \item[Stichwortverzeichnis] W�rter f�r das Verzeichnis m�ssen mit \texttt{index} bekannt gemacht werden \index{Index}. F�r die Zusammenstellung des Verzeichnisses muss im \LaTeX~Editor \texttt{MakeIndex} durchgef�hrt werden.
            \image{content/image/index}{scale=.5}[Index erstellen]
        \end{description}
    %
    %
    %Diverses
    \section{Diverses}\label{s:diverses}
        %
        %TODOs
        \subsection{TODOs}\label{ss:todos}
            Mit dem Befehl \texttt{todo} k�nnen noch unfertige Teile \todo{Noch nicht fertig} markiert werden.