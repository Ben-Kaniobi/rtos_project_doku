%%%%%%%%%%%%%%%%%%%%%%%%%%%%%%%%%%%%%%%%%%%%%%%%%%%%%%%%%%%%%%%%%%%%%%%%%%%%%%%
% Titel:   Howto
% Autor:   Simon Plattner
% Datum:   24.04.2014
% Version: 0.0.1
%%%%%%%%%%%%%%%%%%%%%%%%%%%%%%%%%%%%%%%%%%%%%%%%%%%%%%%%%%%%%%%%%%%%%%%%%%%%%%%
%
%:::Change-Log:::
% Versionierung erfolgt auf folgende Gegebenheiten: -1. Release Versionen
%                                                   -2. Neue Kapitel
%                                                   -3. Fehlerkorrekturen
%
% 0.0.1       Erstellung der Datei
%%%%%%%%%%%%%%%%%%%%%%%%%%%%%%%%%%%%%%%%%%%%%%%%%%%%%%%%%%%%%%%%%%%%%%%%%%%%%%%    
\chapter{Roboterarme}\label{ch:roboterarme}
    Die Ansteuerung der Roboterarme erfolgt �ber einen eigenen Task. Pro Arm wird dazu ein entsprechender Task erzeugt. Das bietet den Vorteil, dass beide Arme weitestgehend unabh�ngig voneinander arbeiten k�nnen. Diese Flexibilit�t k�nnte sich als Vorteil herausstellen, falls das Projekt zu einem sp�teren Zeitpunkt erweitert werden w�rde.\par
    %
    Abbildung \ref{abb:robot_arm_left} zeigt den linken Roboterarm. Der rechte ist nahezu identisch. Die Arme unterscheiden sich in der Mechanik leicht voneinander. Dieser kleine Unterschied kann auf Fertigungstoleranzen zur�ckgef�hrt werden. Die einzige f�r das Projekt relevante Konsequenz ist, dass beide Arme nach dem gleichen Befehl leicht verschiedene Positionen anfahren.
    %    
    \image{appendix/StellarisRoboterModell}{page=11,trim=2cm 16.7cm 11cm 4.5cm,clip=true,width=0.5\textwidth}[Linker Roboterarm][abb:robot_arm_left]
    %
    \image{content/UML/activity_robot_arms}{width=1.0\textwidth}[Aktivit�tsdiagramm \textit{Roboterarme}][abb:activity_robot_arms]
    %
    Abbildung \ref{abb:activity_robot_arms} zeigt die Funktionsweise der Arme auf. Zu beachten ist dabei, dass die Roboterarme sowohl gegenseitig, als auch zwischen sich und den F�rderb�ndern, eine Synchronisation mittels Semaphoren vorsehen. Dabei enthalten die Semaphoren jeweils eine Information �ber den Zustand der Anlage. Sie sch�tzen dadurch faktisch einen Teil des Raumes vor mehrfacher Nutzung (zum Beispiel zwei Roboterarme am gleichen Ort).\par
    %
    Um die \textit{Gesten} der Roboter m�glichst einfach einstellen zu k�nnen, wurde ein einfaches kinematisches Modell der Roboter erstellt. Auf diese Weise kann eine vorgegebene \textit{Hand}-Stellung in die zugeh�rigen Gelenk-Winkel �berf�hrt werden. Aufgrund der Anzahl sowie Anordnung der Gelenke kommt dieses inverse Kinematik-Modell mit analytischen Berechnungen aus - es sind keine (numerischen) N�herungsl�sungen n�tig.