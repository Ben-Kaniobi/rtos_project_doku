%%%%%%%%%%%%%%%%%%%%%%%%%%%%%%%%%%%%%%%%%%%%%%%%%%%%%%%%%%%%%%%%%%%%%%%%%%%%%%%
% Titel:   F�rderb�nder
% Autor:   kasen1
% Datum:   2014-06-07
% Version: 0.0.1
%%%%%%%%%%%%%%%%%%%%%%%%%%%%%%%%%%%%%%%%%%%%%%%%%%%%%%%%%%%%%%%%%%%%%%%%%%%%%%%
%
%:::Change-Log:::
% Versionierung erfolgt auf folgende Gegebenheiten: -1. Release Versionen
%                                                   -2. Neue Kapitel
%                                                   -3. Fehlerkorrekturen
%
% 0.0.0       Erstellung der Datei
%%%%%%%%%%%%%%%%%%%%%%%%%%%%%%%%%%%%%%%%%%%%%%%%%%%%%%%%%%%%%%%%%%%%%%%%%%%%%%%    
\chapter{F�rderb�nder}\label{ch:foerderbaender}
Das Model verf�gt �ber drei F�rderb�nder, die jeweils �ber zwei Lichtschranken verf�gen. In \autoref{abb:foerderband_links} ist das linke F�rderband abgebildet. Das F�rderband rechts ist gleich aufgebaut und einfach um \SI{180}{\degree} gedreht, dies gilt auch f�r die Anordnung der Lichtschranken. Lediglich der Motor und die Elektronik befinden sich aus praktischen Gr�nden auf der anderen Seite was aber die Ansteuerung nicht beeinflusst. Beim F�rderband in der Mitte befinden sich die Lichtschranken nicht am Anfang sondern kurz vor dem Verteiler ("`Flipper"').

\image{appendix/StellarisRoboterModell}{page=9,trim=2.2cm 19.4cm 10.7cm 6.2cm,clip=true,width=0.5\textwidth}[F�rderband Links][abb:foerderband_links]

F�r die Ansteuerung des linken und rechten F�rderbands wird je ein eigener Task verwendet, damit der parallele Ablauf einfach behandelt werden kann. Der Taskablauf ist jedoch bei beiden Tasks der selbe; dieser ist in \autoref{abb:activity_foerderband} ersichtlich. Das mittleren F�rderbands wird im Flipper-Task angesteuert und ist somit in \autoref{ch:flipper} beschrieben.

\image{content/UML/activity_conveyor}{scale=1}[Aktivit�tsdiagramm der F�rderband-Tasks][abb:activity_foerderband]