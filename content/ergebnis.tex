%%%%%%%%%%%%%%%%%%%%%%%%%%%%%%%%%%%%%%%%%%%%%%%%%%%%%%%%%%%%%%%%%%%%%%%%%%%%%%%
% Titel:   Ergebnis
% Autor:   
% Datum:   
% Version: 0.0.0
%%%%%%%%%%%%%%%%%%%%%%%%%%%%%%%%%%%%%%%%%%%%%%%%%%%%%%%%%%%%%%%%%%%%%%%%%%%%%%%
%
%:::Change-Log:::
% Versionierung erfolgt auf folgende Gegebenheiten: -1. Release Versionen
%                                                   -2. Neue Kapitel
%                                                   -3. Fehlerkorrekturen
%
% 0.0.0       Erstellung der Datei
%%%%%%%%%%%%%%%%%%%%%%%%%%%%%%%%%%%%%%%%%%%%%%%%%%%%%%%%%%%%%%%%%%%%%%%%%%%%%%%    
\chapter{Ergebnis}\label{ch:ergebnis}
Unser Ziel war es, einen m�glichst schnellen Kreislauf der ECTS zu erm�glichen. Das sollte mittels verschiedener Optimierungen\footnote{Optimierungen gegen�ber der Beispiell�sung; der Quellcode dieser anderen L�sung stand nicht zur Verf�gung.} erreicht werden. Vorgesehen war zu diesem Zweck das beschleunigen der Roboterarme. Sie sollten m�glichst fl�ssig handeln und Wartezeiten vermeiden.\par
%
Das gesteckte Ziel konnte nicht erreicht werden. Grund daf�r ist die Hardware - sie ist wesentlich langsamer als angenommen. Beispielsweise f�hrt der Roboter nur maximal etwa 2-3 Befehle pro Sekunde aus. F�r eine fl�ssige Bewegung ist das deutlich zu wenig. Weiterhin schwingen die Arme nach jeder Bewegung nach. Da eine R�ckkoppelung nicht gegeben ist, kann dieses Verhalten nicht auf vern�nftige Weise kompensiert werden.\par
%
W�hrend unserer Tests haben sich einige Hardwarekomponenten anders verhalten als gem�ss Handbuch zu erwarten w�re\footnote{Insbesondere der Totalausfall einer der beiden Arme kam unerwartet.}. Dieser Sachverhalt wurde mit dem Dozenten besprochen.\par
%
Im Bezug auf das verwendete Echtzeitbetriebssystem konnte eine funktionierende L�sung entwickelt und implementiert werden. Das System l�uft dank der umgesetzten Task-Synchronisation stabil\footnote{Das System wurde aus Zeitgr�nden nicht ausgiebig getestet. Bei den durchgef�hrten Tests konnte jedoch nie ein Synchronisationsproblem oder ein Deadlock beobachtet werden.}.