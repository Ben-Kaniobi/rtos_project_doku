%%%%%%%%%%%%%%%%%%%%%%%%%%%%%%%%%%%%%%%%%%%%%%%%%%%%%%%%%%%%%%%%%%%%%%%%%%%%%%%
% Titel:   ECTS-Updater
% Autor:   kasen1
% Datum:   2014-06-07
% Version: 0.0.1
%%%%%%%%%%%%%%%%%%%%%%%%%%%%%%%%%%%%%%%%%%%%%%%%%%%%%%%%%%%%%%%%%%%%%%%%%%%%%%%
%
%:::Change-Log:::
% Versionierung erfolgt auf folgende Gegebenheiten: -1. Release Versionen
%                                                   -2. Neue Kapitel
%                                                   -3. Fehlerkorrekturen
%
% 0.0.0       Erstellung der Datei
%%%%%%%%%%%%%%%%%%%%%%%%%%%%%%%%%%%%%%%%%%%%%%%%%%%%%%%%%%%%%%%%%%%%%%%%%%%%%%%    
\chapter{ECTS-Updater}\label{ch:ectsupdater}
Damit der Zustand aller ECTS-Steine zu jeder Zeit abgefragt werden kann (\zB\ f�r die graphische Darstellung), wurde ein Task entworfen der diese Informationen regelm�ssig aktualisiert. Abgelegt sind diese Informationen jeweils in einer Struktur die folgende Daten enth�lt:

\begin{description}[leftmargin=!,labelwidth=\widthof{\bfseries id (uint8\_t):}]
	\item[id (\textit{uint8\_t}):] Die ID des ECTS zur Identifikation
	\item[x (\textit{uint16\_t}):] Die x-Position des ECTS, sie entspricht der zur�ckgelegten Distanz auf dem jeweiligen F�rderband in Millimeter
	\item[y (\textit{uint16\_t}):] Die y-Position des ECTS, sie entspricht der links-rechts-Verschiebung auf dem Fliessband (aufgeteilt in 8 "`Bahnen"')
	\item[z (\textit{z\_pos}):] Das Teilsystem bei dem sich das ECTS befindet
\end{description}

\vspace{-12pt}
\image{content/UML/xyz}{scale=.7, trim=0 1cm 0 0}[�bersicht �ber die Positionsdaten][abb:xyz]
\vspace{-12pt}

In \autoref{abb:ectsupdater} ist der Taskablauf ersichtlich. Die x- und y-Positionen aller ECTS-Strukturen werden mit Informationen von Lichtschranken aktualisiert. Zus�tzlich wird die x-Position regelm�ssig mit einem von der F�rderband-Geschwindigkeit abh�ngigen Wert inkrementiert. Beim Wechseln in ein anderes Teilsystem wird die x-Position wieder auf 0 gesetzt und die z-Position aktualisiert.

\image{content/UML/activity_ects_updater}{scale=.5, trim=0 1cm 0 0}[Aktivit�tsdiagramm des ECTS-Updater-Tasks][abb:ectsupdater]
\vspace{-20pt}
Zur Vereinfachung des Taskaufbaus und damit auch andere Tasks einfach die ECTS-Informationen aktualisieren k�nnen (beispielsweise beim Wechsel vom F�rderband zum Roboterarm), wurden Hilfsfunktionen entwickelt. Diese werden nachfolgend kurz erkl�rt.

\section{Funktion \texttt{CAN\_conveyor\_status\_handler}}\label{s:canconveyorstatushandler}
Die Funktion \texttt{CAN\_conveyor\_status\_handler} wird aufgerufen, nachdem per CAN eine Antwort auf die Statusanfrage der Lichtschranken empfangen wurde. In ihr werden die Daten der Antwort verarbeitet und damit die ECTS-Strukturen aktualisiert.

\section{Funktion \texttt{find\_ECTS}}\label{s:findects}
Wenn eine ECTS-Struktur mit Informationen aus dem System aktualisiert werden soll muss vorerst die richtige ECTS-Struktur gefunden werden, denn die ECTS-Steine unterscheiden sich nur durch die Position. F�r diese Aufgabe wurde die Funktion \texttt{find\_ECTS} implementiert. Als Eingangsparameter wird die z-Position angegeben in der Sich das ECTS-Befindet. Die Funktion liefert liefert dann einen Zeiger auf die gefundene ECTS-Struktur an dieser z-Position zur�ck. Wenn sich mehrere ECTS an dieser Stelle befinden (bei den F�rderb�ndern), so wird ein Zeiger auf die ECTS-Struktur mit der gr�ssten x-Position zur�ckgegeben.

